\documentclass{standalone}
\usepackage{amsmath}
\usepackage{xcolor}

\newcommand{\binomialformula}[3]{
    P(X = #2) = \binom{#1}{#2} (#3)^{#2} (1-#3)^{#1-#2}
}

\newcommand{\hipergeometricaformula}[4]{
    P(X = #4) = \frac{\binom{#2}{#4} \binom{#1-#2}{#3-#4}}{\binom{#1}{#3}}
}

\newcommand{\bernoulliformula}[2]{
    P(X = #2) = (#1)^{#2} (1-#1)^{1-#2}
}

\newcommand{\geometricaformula}[2]{
    P(X = #2) = (1-#1)^{#2-1} #1
}

\newcommand{\poissonformula}[2]{
    P(X = #2) = \frac{#1^{#2} e^{-#1}}{#2!}
}

\begin{document}

% Ejemplo de uso:
% Binomial
\[\binomialformula{10}{4}{0.5}\]

% Hipergeométrica
\[\hipergeometricaformula{50}{20}{10}{4}\]

% Bernoulli
\[\bernoulliformula{0.5}{1}\]

% Geométrica
\[\geometricaformula{0.3}{5}\]

% Poisson
\[\poissonformula{2.5}{3}\]

\end{document}